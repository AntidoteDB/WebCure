% Abstract for the TUM report document
% Included by MAIN.TEX

\begin{center}
{\Large \textbf Abstract}
\end{center}
\vspace{1cm}

The goal of this thesis is to examine the possibilities of developing an offline web applicaiton, 
which would serve as a client for the AntidoteDB database. A prototype of the application is developed. 
The architecture of the application is designed in such a way that both offline and online functionality are possible. 
Conflict-free Replicated Data Types (CRDTs) are used for modeling the data to be stored offline in the local database of 
a web-browser. It lets to ease the task of merging the changes. 

Apart from that, the client-server protocol was designed, in order to support the functionality of the application. 
The paper could be logically divided into two parts: firstly, the problem of designing mentioned solution is going
 to be discussed. Secondly, there is an implementation part and description how specific problems were tackled.