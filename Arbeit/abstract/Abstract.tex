\begin{center}
{\Large \textbf Abstract}
\end{center}
\vspace{1cm}

Despite advancements in connectivity for user devices by service providers, such devices are subject to periods of disconnection. In order for the user to be able to interact with the application during periods of disconnectivity, the application must store its data on the client machine. Besides that, updates need to be maintained and delivered to the server once the connection is reestablished. The demand for offline support has led to many ad-hoc solutions that often do not provide well-defined consistency guarantees. 

In this thesis, we developed a WebCure, a framework for partial replication of data at client-side in web applications. It consists of a client-side data store that maintains both the data that has been received from the cloud storage server and the updates that have been executed by the client but have not been delivered to the cloud store, yet. A service worker acts as a proxy on client-side. While the client is offline, it forwards the requests to the client-side database. Finally, a cloud storage server maintains data shared between different clients. For this purpose, we chose AntidoteDB as the cloud storage server since it provides with conflict-free replicated data types (CRDTs) a well-defined semantics for concurrent updates. Updates that are executed while a client is offline are concurrent with all other updates happening between the last retrieval from the cloud storage and the next connection and synchronisation of the client. We developed the algorithms for WebCure and implemented a stable prototype of the framework. To evaluate its feasibility and performance, we additionally ported a calendar application that allows now for offline operations.