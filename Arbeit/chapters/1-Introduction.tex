\chapter{Introduction}
\label{Introduction}

In this chapter we are going to discuss the motivation, research questions and the scope of this thesis.

\section{Motivation}

AntidoteDB is a planet scale, highly available, transactional database. It accepts parallel updates that are geo-distributed near users. As well as that, it is available under partition, efficient and responsive. At the same time, AntidoteDB avoides anomalies, thanks to the following features: it merges concurrent operations according to their semantics, it groups related operations into atomic transactions, delivers updates in causal order. The main motivation of this thesis is to explore the possibillities of implementing a web-client for the AntidoteDB with a support of caching and by utilizing the main features of the database. As AntidoteDB is already known for its use in the development of applications, it would dramatically improve the user-experience if offline work with the database is going to be reached. I am going to explore the best ways to develop such an application, design it and implement. Here and from now on, we will call such an application by the name of WebCache.


The motivation of this thesis is to show you the functionalities of AntidoteDB. Therefore, it is not to demonstrate some brand new-idea in terms of application. But it is to show that it is possible to build an application based on Antidote-DB that works offline and online.

%Nowadays, there are many fields, where distributed databases are used. The scenarios could be different, but to name some of them, it could be booking tickets for the flights and concerts, watching video streaming services (e.g. Netflix) and other examples. There are two main types of the distributed databases: strongly consisted and eventually consistent. However, each of those has its own drawbacks. Thus,   
%(Antidote is a geo-distributed cloud database that is highly responsive and available under partition, it offers strong guarantees to applications. )
% A database side is strongly consistent, though. 

\section{Research questions}

The following research questions are going to be addressed in this thesis:

\begin{itemize}
    \item \textbf{RQ1.} How efficient is it to use a web-client with cache rather than without it?
    \item \textbf{RQ2.} What are the methods available to implement web-applications that would be able to work off-line and
    in the conditions of poor network connections?
    \item \textbf{RQ3.} What could be a scalable solution for transmitting CRDT data between a server and clients?
  \end{itemize}





\section{The structure}

Here, I am going to briefly sketch the structure of this thesis and briefly explain the contents of each chapter. 

\begin{itemize}
    \item Description of the main requirements of the application;
    \item Design of the architecture 
    \item Description of the implementation phase
    \item Evaluation
    \item Conclusion
  \end{itemize}