\chapter{Introduction}
\label{Introduction}

In this chapter we are going to discuss the motivation, research questions and the scope of this thesis.

o Problem context (What is the broader context of this work?) 
o Problem description (What is the specific problem or challenge addressed by this work?) 
o Research question(s) (What are the actual questions that should be answered by this research? Why are they interesting? To whom? Why is there no obvious answer? If this is getting too much, then details of discussion can be deferred until later, e.g., to the methodology section.) 
o Summary of results and contributions (What is the outcome of this work in simple terms?) 
o Structure of the thesis (Introduce briefly the remaining chapters.)

\section{Motivation}

The motivation of this thesis is to explore the possibillities of implementing 
a web-client for the AntidoteDB with a support of caching and by utilizing the main features of the AntidoteDB.

\section{Research questions}

The following research questions are going to be addressed in this thesis:

\begin{itemize}
    \item \textbf{RQ1.} How efficient is it to use a web-client with cache rather than without it?
    \item \textbf{RQ2.} What are the methods available to implement web-applications that would be able to work off-line and
    in the conditions of poor network connections?
    \item \textbf{RQ3.} What could be a scalable solution for transmitting CRDT data between a server and clients?
  \end{itemize}





\section{The structure}

The structure of this thesis will be divided into the following subsections: 

\begin{itemize}
    \item Description of the main requirements of the application;
    \item Design of the architecture 
    \item Description of the implementation phase
    \item Evaluation
    \item Conclusion
  \end{itemize}