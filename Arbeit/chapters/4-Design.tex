\chapter{Design}
\label{Design}

In this chapter, we are going to, firstly, introduce the requirements for the WebCure. Having set them, afterwards, we will in details describe the design of the solution.

\section{Requirements}
\label{Requirements}

We listed the functional requirements of the WebCure at \tableref*{table:req1} and non-functional requirements at \tableref*{table:req2}.

\begin{table}[!htb]
\centering
\caption{Functional requirements.}
\label{table:req1}
\begin{tabular}{|p{1cm}|p{12cm}|}
\hline
R1 & Retrieval, increment and decrement of the Counter CRDT should be possible.                         \\ \hline
R2 & Retrieval, adding and removing elemets from the Set CRDT should be possible.                       \\ \hline
R3 & Retrieval, assigning and resetting the Multi-Value Register CRDT should be possible.               \\ \hline
R4 & Retrieval elements of any supported CRDTs should be possible according to the passed timestamp. \\ \hline
R5 & Do not store the element's values in the cache, if they were requested at specific timestamp (for the matter of having only latest data on the client's side). \\ \hline
R6 & The user should be able to remove from the client storage any element of supported CRDTs. \\ \hline
R7 & It should not be possible to create elements of different CRDTs with the same name (due to described limitations of AntidoteDB) \\ \hline
R8 & When offline, it should be possible to perform operations on supported CRDTs. \\ \hline
R9 & Any operations performed offline, once the connection is restored, should be sent to the server immediately. \\ \hline
R10 & When the connection is re-established after having data changes in offline mode, the client data should be updated with relation to current client's changes, as well as with relation to the changes made on the server. \\ \hline
\end{tabular}
\caption*{}
\end{table}

\begin{table}[!htb]
\centering
\caption{Non-functional requirements.}
\label{table:req2}
\begin{tabular}{|p{1cm}|p{12cm}|}
\hline
NFR1 & The web application should be available online and offline (except of the functionality with timestamp-related updates). \\ \hline
NFR2 & The web application should be available with a poor internet connection. \\ \hline
\end{tabular}
\caption*{}
\end{table}

As we specified the requirements, we can go further into the design and design the protocol of the system.






























\section{Overview of the protocol}

The fundamental part of the WebCure will be its protocol design. We need to examine the following cases, in order to specify the protocol of the application:

    \begin{itemize}
        \item {A client receives an update from the server}
        \item {A client sends an update to the server}
        \item {Two clients interact with a server}
    \end{itemize}

\subsection{The way of exchaning the data}

As we already know from the \chapref*{Background}, because the AntidoteDB is using CRDT datatypes, the following options are possible to update the database: state-based and operation-based. Due to the time constraints and the amount of work, this thesis will consider only the operation-based approach. Therefore, whenever a client wants to update the database, it will send to the server a list of operations. However, whenever it wants to read the value, it will receive the current state of the object from the database. For this thesis, we are going to use such datatypes as Counters, Sets and Multi-Value Registers, to which the reader was introduced in the \chapref*{Introduction}.

\subsection{A client receives an update from the server}

Let us say that a client requests an update from the server. In this case, if the request is successful, the server is going to respond with a value for the requested key and the timestamp of the last write -- \textit{t\textsubscript{0}}.

\begin{figure}[!htb]
    \begin{center}
    \def\svgwidth{\linewidth}
    \input{images/architecture/read.pdf_tex}
    \caption {The way a client receives a key-value update from the server.}
    \label{fig:design2}
\end{center}
\end{figure}

In case a client receives an update from the server, and that update has an earlier timestamp than the one, which is already stored in the client's cache, then a client skips it and tries later for fresher updates. The implementation of this will be explained later in the chapter of implementation.

\subsection{A client sends an update to the server}

In the case of writing the information to the server, a client has to send a key with an operation to the server. Then the acknowledgment with a timestamp is going to be sent back to the client.

\begin{figure}[!htb]
    \begin{center}
    \def\svgwidth{\linewidth}
    \input{images/architecture/write.pdf_tex}
    \caption {The way a client sends a write operation to the server.}
    \label{fig:design3}
\end{center}
\end{figure}

\subsection*{Several updates on a client while offline}

The client is capable of performing updates when offline. These updates will take effect immediately. However, in order for them to be applied to the server, the client has to be back online. Once the connection is established again, all the updates that were performed on the client-side in offline mode will be sent to the server.

\subsection{Two clients interact with a server.}

\begin{figure}[!htb]
    \begin{center}
    \def\svgwidth{\linewidth}
    \input{images/architecture/twoClients.pdf_tex}
    \caption {Graphical representation over a possible communicaton between two clients and a server.}
    \label{fig:design4}
\end{center}
\end{figure}

Let us assume that initially, the server has the key-value pair \textit{(k, v)} at the timestamp \textit{t\textsubscript{0}}. Therefore, after both clients receive updates from the server, they consist of the \textit{(k, v)} pair at the timestamp \textit{t\textsubscript{0}}. In case clients change the value under mentioned key to something else, they will have to get an acknowledgement from the server in order to receive a unique timestamp related to the change. At the representation above, a \textit{Client 1} is acting first and getting an acknowledgement of its change at the time \textit{t\textsubscript{1}}, while Client \textit{Client 2} makes the change later at time \textit{t\textsubscript{2}}. That makes \textit{Client 1} receive a new value \textit{v''}, when it reads the information from the server again.

\section{Offline functionality}

In this section, we are going to describe the offline functionality of the system.

Initially, the database is empty. Therefore, if the user is offline from the very beginning, he should be able to add the data into the database himself. 
The system represented in \figref*{fig:design5} will change by having only the Web Application and the database. However, whenever the connection is established with the server, the operations, which were stored in local database while offline, will be sent to the server. 

At the \figref*{fig:design6}, the sequence of getting the data from the local database is shown. This case describes the scenario, when the server is unavailable and the application has to read the value from the local database. 

\begin{figure}[!htb]
    \begin{center}
    \def\svgwidth{\columnwidth}
    \input{images/design/offlineProtocol.pdf_tex}
    \caption {Successful request of state from the local database -- Sequence diagram.}
    \label{fig:design6}
\end{center}
\end{figure}


\begin{figure}[!htb]
    \begin{center}
    \def\svgwidth{\columnwidth}
    \input{images/design/offlineStore.pdf_tex}
    \caption {Successful storing of an operation in the local database -- Sequence diagram.}
    \label{fig:design7}
\end{center}
\end{figure}

At the \textbf{Figure \ref{fig:design7}}, the sequence of storing the data locally is shown. When the connection is not there, the application will store in the database all offline-performed operations by the user. Afterwards, once the connection is re-established, that data will be sent back to the server. At the point when we read the state again from the server, the data, which is stored locally could be easily removed due to be pointless to hold on it. 


\section {Online functionality}

In this section, we are going to describe the online functionality of the system.



\section {The transition between offline and online modes}