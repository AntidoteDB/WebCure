\chapter{Design}
\label{Design}

In this chapter, we are going to, firstly, introduce the requirements and assumptions we make for the design of WebCure. Having set them, afterwards, we will in detail describe the design of the system.

\section{Requirements}
\label{4-Requirements}

We listed the functional requirements of WebCure in \tableref*{table:req1} and non-functional requirements in \tableref*{table:req2}.

\begin{table}[!htbp]
\centering
\caption{Functional requirements.}
\label{table:req1}
\begin{tabular}{|p{1cm}|p{14cm}|}
\hline
R1 & Retrieval, increment and decrement of the Counter CRDT should be possible.                         \\ \hline
R2 & Retrieval, adding and removing elements from the Set CRDT should be possible.                       \\ \hline
R3 & Retrieval, assigning and resetting the Multi-Value Register CRDT should be possible.               \\ \hline
R4 & Retrieval elements of any supported CRDTs should be possible according to the passed timestamp. \\ \hline
R5 & The client should cache only the latest data available at server's side. \\ \hline
R6 & It should not be possible to create elements of different CRDTs with the same name (due to limitations of AntidoteDB). \\ \hline
R7 & When offline, it should be possible to make read/write operations on supported CRDTs. \\ \hline
R8 & The user should be able to remove from the client any stored data element. \\ \hline
R9 & Any operations performed offline, once the connection is restored, should be sent to the server immediately. \\ \hline
R10 & The execution model of offline operations at the client should be sequential (updates are ordered). \\ \hline 
R11 & When the connection is reestablished after having data changes in offline mode, the client storage should be updated appropriately (with a consideration of the client's offline changes and possible changes on the server). \\ \hline
\end{tabular}
\caption*{}
\end{table}

\begin{table}[!htbp]
\centering
\caption{Non-functional requirements.}
\label{table:req2}
\begin{tabular}{|p{1cm}|p{14cm}|}
\hline
NFR1 & The system should be available online and offline (except for the functionality with timestamp-related updates). \\ \hline
NFR2 & The system should be available with a poor internet connection. \\ \hline
\end{tabular}
\caption*{}
\end{table} 

First of all, looking at the first three functional requirements, \textit{R1--R3}, we see that for the implementation were selected such CRDTs as Counter, Set and Multi-Value Register. The reason for that is that these data types cover the most operations on CRDTs and, if our design works for them, it will work for the rest as well. The requirement \textit{R4} makes the user able to get the data from the server at different timestamps and compare it. Next, talking about the requirement \textit{R5}, even though it is possible to request data at different timestamps, it is vital for the client to store in cache only the latest available at the server data. The requirement \textit{R6} is due to one of the limitations AntidoteDB possesses, and is related to one of the assumptions we make in the following section. Requirements \textit{R7--R9} specify what kind of actions the user is able to perform on a client when offline, as well as their synchronisation with the server. Finally, the requirements \textit{R10} and \textit{R11} guarantee that the changes performed offline at the client, as well as changes at the server side, are synchronised in a way satisfying causal consistency model. 

Next, there are non-functional requirements: \textit{NFR1} and \textit{NFR2}. Basically, both of them support our claim that a client is able to work offline, as well as under uncertain network conditions. 

\section{Assumptions}
\label{4-Assumptions}

In this section, we are going to give a list of assumptions we make for WebCure. 

\begin{enumerate}
\item {\textbf{Timestamps}. Firstly, the database storage used for the server's side should have the concept of timestamps (like in AntidoteDB, described in \secref*{2-antidotedb}), in order for the protocol we are going to describe in the \secref*{4-protocol} to work correctly.}
\item {\textbf{Cache is persistent}. For WebCure to work online and, especially, offline, we believe that the browser's cache is safe from automatic clearing. Contrarily, if the cache could be cleared automatically depending on the browser's behaviour, it makes it impossible to support the claim that the application can work offline. To guarantee this assumption, a Persistent Storage API described in \secref*{persistentstorage} can be used.}
\item{\textbf{Name duplicates}. We limit the creation of different CRDT elements with the same name in the system due to limitations of AntidoteDB, as the requirement \textit{R7} describes it in the \tableref*{table:req1}. As AntidoteDB is in the process of ongoing development, currently the database crashes when there is an attempt to create elements of different CRDTs with the same name. Thus, this condition has to be fulfilled.}
\item{\textbf{Server's database is always on}. We assume that the server's database is not going to be reset and lose all its data. As the client entirely relies on the server's storage for the synchronisation and only sends back operations performed offline on client's side, it will be impossible to restore the server's database from the client's storage, even if it was up-to-date before the server's data loss. With additional changes to the current protocol, it might be possible, though, but that is not the topic we cover in this thesis. However, even in such a situation, the client will be able to continue the offline work.}
\end{enumerate}

As we specified the requirements, we can go further into and design the protocol of the system.

\section{Protocol}
\label{4-protocol}

The fundamental part of WebCure will be its protocol design. We are going to describe it in an event-based way in the form of pseudo-code in the following sections. 

%    \begin{itemize}
   %     \item {A client receives an update from the server}
      %  \item {A client sends an update to the server}
       % \item {Two clients interact with a server}
    %\end{itemize}

\subsection{Data transmission}

As we already remember from the \chapref*{Background}, because AntidoteDB is using CRDT datatypes, the following options are possible to update the database: state-based and operation-based. This thesis will consider only the operation-based approach, as it has such an advantage over the state-based approach as less transferred data in most situations. Therefore, whenever a client wants to update the database, it will send to the server a list of operations. However, whenever it wants to read the value, it will receive the current state of the object from the database. For this thesis, we are going to use such datatypes as Counters, Sets and Multi-Value Registers, to which the reader was introduced in the \secref*{2-crdts}.

\subsection{Description} 

\begin{figure}[!htb]
    \begin{center}
    \def\svgwidth{0.6\linewidth}
    \input{images/protocol/overview.pdf_tex}
    \caption {An overview of the communication protocol.}
    \label{fig:protocol1}
\end{center}
\end{figure}

Firstly, as we would like to focus on the communication part between a server and a client, let us for now keep both of them as black boxes\footnote{In software engineering, a black box is a system, which can be viewed in terms of its inputs and outputs, without the understanding of its internal workings.\cite{49}}, as they are represented at \figref*{fig:protocol1}. Next, we will go through different stages of their communication and describe, how we handled these processes. 

\subsection*{Graphics notations}

\begin{figure}[!htb]
    \centering
    \def\svgwidth{0.35\linewidth}
    \subfloat[]{{\input{images/notations/timeline.pdf_tex}}}%
    \qquad
    \def\svgwidth{0.35\linewidth}
    \subfloat[]{{\input{images/notations/transmission.pdf_tex}}}%
 \def\svgwidth{0.35\linewidth}
    \subfloat[]{{\input{images/notations/data.pdf_tex}}}%
    \qquad
 \def\svgwidth{0.35\linewidth}
    \subfloat[]{{\input{images/notations/operation.pdf_tex}}}%
    \qquad
    \caption{An overview of notations used in the following chapters for the protocol explanation.}%
    \label{fig:notations}%
\end{figure}

Let us explain the notations, which are going to be used for a further protocol description. At \figref*{fig:notations} \textbf{(a)}, we can see a notation for the timeline. Timelines will be used for the matter of showing the sequence of events happening. At \figref*{fig:notations} \textbf{(b)}, the arrow shows the transmission of data between a subsystem \textit{A} and subsystem \textit{B}, as well as its direction and a command. \figref*{fig:notations} \textbf{(c)} represents the state of the data on a system's side, while \figref*{fig:notations} \textbf{(d)} points to a timestamp, at which an operation that changes the system's storage was applied.

Next, as we already mentioned, we will explain the protocol in an event-based way. 

\subsection*{A client receives an update from the server}

\begin{figure}[!htb]
    \begin{center}
    \def\svgwidth{\linewidth}
    \input{images/architecture/read.pdf_tex}
    \caption {The communication between a client and a server for the read function.}
    \label{fig:design2}
\end{center}
\end{figure}

Let us assume that a client initiates its work with an empty storage. Then, a user might want to request the actual data from the server. In this case, as can be seen at \figref*{fig:design2}, a user has to pass to the server an \textit{id} of the data to read. If the request is successful, the server is going to respond with a \textit{value} for the requested \textit{id} and the timestamp of the last write -- \textit{t\textsubscript{0}}, so the client will store this information in its own storage.

\begin{lstlisting}[caption={Pseudocode for requesting the data: client.}, label={lst:read1}]
// Read function that pulls database changes
// @param id: the id of the object, for which the update was requested;

function read(id) {
  GetHttpRequest( // send an http-request to get the data from the server by id
    id,
    function onSuccess(value, timestamp) { // get the value and timestamp from the server

      // create an object that maintains all necessary data
      var item = {
        id: id, // id of an data item
        operations: [], // operations performed offline
        sentOperations: [], // operations performed offline and already sent to the server
        state: value, // a state received from the server
        type: 'set' // a type of CRDT
      }; 

      storeInCache(item); // cache an item created earleir
      storeInCache(timestamp); // cache the timestamp
    },
    function onFail() {
      getFromCache(id); // get the object from cache by id
    }
  );
}
\end{lstlisting} 

The pseudocode of the logic for this \textit{read} functionality can be seen at the \lstref*{lst:read1}. At the \textit{line 5}, an asynchonous function \textit{GetHttpRequest} has two callbacks -- \textit{onSuccess}, in case a request proceeds successfully, and \textit{onFail} in case of a failure. 

In case of success, the value and the timestamp associated with passed \textit{id} are going to be fetched from the server. After that, at the \textit{line 10} we create an element \textit{item}, which has the following properties: \textit{id} for the id of a data item, \textit{state} for the actual state of the data item on the server, \textit{type} for the type of CRDT, \textit{operations} for the operations performed at the client's side while offline, \textit{sentOperations} for the operations performed at the client's side while offline, but which are already sent to the server. Next, a function \textit{storeInCache} is called twice at the lines \textit{18 and 19}. Firstly, to store in cache an object \textit{item} for the offline use. Secondly, to store in cache a \textit{timestamp} received from the server.  A client will always receive either the data associated with the latest timestamp from the server or, if a client chooses to specify the timestamp, it will receive the data associated with that timestamp. 

In case of failure, however, the method \textit{getFromCache} is going to be called with an argument \textit{id}, as can be seen at the \textit{line 22}. If \textit{id} exists at client's cache, then it will be returned. 


Now, let us say, that after receiving an element \textit{id} from the server, the client wants to change it and send it back to the server.

\subsection*{A client sends an update to the server}

\begin{figure}[!htb]
    \begin{center}
    \def\svgwidth{\linewidth}
    \input{images/architecture/write.pdf_tex}
    \caption {The communication between a client and a server for the update function.}
    \label{fig:design3}
\end{center}
\end{figure}

Looking at the \figref*{fig:design3}, in the case of writing the information to the server, a client has to send an \textit{id} with an operation to perform. After that, the server is going to apply the received operation on its side and, in case of success, the new state of the data will receive a timestamp \textit{t\textsubscript{1}}, and an acknowledgement of the successful commit will be sent back to the client. What happens in case of unsuccessful acknowledgement will be explained below. 

So, once the client is notified that the update was applied on the server successfully, a user can get the latest changes to the client's side now. Thus, when the read request for \textit{id} is sent again, the server will send back the new value -- \textit{value'} and a new timestamp -- \textit{t\textsubscript{1}}, for the element \textit{id}.

\begin{lstlisting}[caption={Pseudocode for making a request to change the data: client.}, label={lst:update1}]
// Update function that processes user-made update
// @param id: an id for the object that should be updated;
// @param op: operation performed on the object for the specified id

function update(id, op) {
  PostHttpRequest(
    // send an http-request to update the data on the server
    { id, op },
    function onSuccess(result) {
      // no actions performed
    },
    function onFail() {
      var item = getFromCache(id); // get the object from cache by id
      item.operations.push(op); // store the operation on a client's side in order to try sending it again later
      storeInCache(item); // cache an updated item
    }
  );
}
\end{lstlisting} 

However, let us look at the pseudocode placed at the \lstref*{lst:update1}. There we can see the function \textit{update}, which has to have an access to the parameters \textit{id} and \textit{op}. There is an attempt to send the operation \textit{op} for the element \textit{id} to the server, by using \textit{PostHttpRequest}. It is asynchronous and has two callbacks -- \textit{onSuccess} and \textit{onFail}, just as before it was explained for the \textit{read} function. If the request succeeds, then the client gets notified about it and no further actions are taken. However, in the case of failure, as we can see at the \textit{line 13}, firstly we are getting the object from the cache by \textit{id}. If it exists, then we add the operation \textit{op} to that object into its property \textit{operations}, and, afterwards, store an updated object in cache again using \textit{storeInCache} at the \textit{line 15}. That makes the update available while the user is offline and gives an opportunity to send the operation again when the connection is back.

Next, let us say that a client loses its network connection, so any updates made from that point onwards will be stored locally.  

\subsection*{Offline behaviour}

\begin{figure}[!htb]
    \begin{center}
    \def\svgwidth{\linewidth}
    \input{images/architecture/offline.pdf_tex}
    \caption {The communication between a client and a server while offline with a transition to online.}
    \label{fig:design4}
\end{center}
\end{figure}

Have a look at the \figref*{fig:design4}, where appropriate markings can clearly distinguish periods when the client was offline and online. The client has an element \textit{id} with a value \textit{value'} at timestamp \textit{t\textsubscript{1}} and then makes a local change applying some operation, which changes the previous value to a new one -- \textit{value''}. Pay attention that a new value does not receive a timestamp assigned to it, while locally: to support the causal consistency claims, the responsibility for assigning timestamps should be taken by the server. Then, after some time, the connection gets back, and the client sends an immediate update message to the server with \textit{id} of an element and the applied operation. The server's side, as was already described above, applies that operation and returns an acknowledgement of success. Eventually, the client sends a \textit{read} message and gets back the \textit{value''} as well as the assigned to it timestamp \textit{t\textsubscript{2}}.

\begin{lstlisting}[caption={Pseudocode for sending offline performed operations to the server: client.}, label={lst:offline}]
// Update function that processes user-made updates performed offline when the connection is restored
// @param offlineOperations: an array that contains all the operations performed on a client's side offline
function synchronize(offlineOperations) {
  if (ONLINE) {
    offlineOperations.forEach(operation => {
      send(operation);
    });
  }
}
\end{lstlisting} 

The logic described above can be seen at the~\lstref*{lst:offline}, which has the function named \textit{synchronize} that should be triggered at the time when the client's side restored a network connection. There, we can see that every operation performed offline is sent once at a time to the server. For causality, the array should be sorted in the order the operations were performed initially.

Now, let us move to the point when more than one client interacts with a server, in order to see how scalable the described protocol is.

\subsection*{Two clients interact with a server.}

\begin{figure}[!htb]
    \begin{center}
    \def\svgwidth{\linewidth}
    \input{images/architecture/twoClients.pdf_tex}
    \caption {The communication between two clients and a server.}
    \label{fig:design5}
\end{center}
\end{figure}

Let us assume that, initially, as can be seen at the \figref*{fig:design5}, the server has a stored element \textit{(id, value)} at the timestamp \textit{t\textsubscript{0}}. Therefore, when both clients request to read the data from the server, they get that data and store it locally. At the representation above, a \textit{Client 1} is acting first and sends an update to the server changing the value of an element \textit{id} to \textit{value'} at \textit{t\textsubscript{1}}. Observe that \textit{Client 1} does not request the latest data from the server and still only has its local changes. In parallel, a \textit{Client 2} makes the change later at time \textit{t\textsubscript{2}}, and an element \textit{id} is now set to \textit{value''}. Then, both clients request the updated data from the server and both receive the actual value of the element \textit{id} at the timestamp \textit{t\textsubscript{2}}, which is \textit{value''}. We would like to stress the point that all systems - the server and both clients end up having the same data.

Now we would like to give a brief overview of the next two chapters: firstly, in the \chapref*{Technologies} we will give a proper introduction into the technologies we used to implement the described protocol and, then, in the \chapref*{Implementation} we will go through its development.