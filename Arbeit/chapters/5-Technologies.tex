\chapter{Technologies}
\label{Technologies}

In this chapter, we describe the technologies we used to implement the design we introduced in \chapref*{Design}. To give an overview, the implementation of the design for WebCure demands the following components:

\begin{itemize}
\item{A service worker, in order to manage requests coming from the client;}
\item{A Cache API to store HTML, CSS, JavaScript files, and any static files\cite{41} to make the application available offline;}
\item{A database to store the data locally;}
\item{Background Sync for deferring the actions conducted offline until the connection is stable.}

\end{itemize}

% TODO What about backend ? Requirements / assumptions!!!

\section{Service Worker}

Service worker\cite{1} is a web worker\footnote{Web Workers make it possible to run a script operation in a background thread separate from the main execution thread of a web application\cite{30}.}, which is a JavaScript file, which lies in the middle, between the web application and network requests. Service worker runs independently from the web-page and has the following characteristics: 

\begin{itemize}
\item{It does not have access to the DOM\footnote{DOM, or \textbf{D}ocument \textbf{O}bject \textbf{M}odel of the page defines HTML elements as objects, as well as their properties, methods, and events.}, however, it has the control over pages (not just a specific one);}
\item{When the application is not running, a service worker still can receive push-notifications from the server\cite{32}, which let us improve the user experience of the application and makes it ``closer'' to native mobile applications;}

\item{It runs over HTTPS, as for the projects using a service worker, the ``man-in-the-middle''\footnote{In computer security, this is a type of attack, where a third party secretly alters the communication between two parties, while they believe they are directly ``talking'' to each other\cite{33}.} attacks could represent a real threat;}
\item{It offers a possibility to intercept requests as the browser makes them and has the following options:}
\begin{itemize}
\item{to let requests go to the network as usual;}
\item{to skip the network and redirect requests to get the data from the cache;}
\item{to perform any combination of the above.}
\end{itemize}
\end{itemize}

\begin{figure}[!htb]
    \begin{center}
    \def\svgwidth{0.7\linewidth}
    \input{images/technologies/sw.pdf_tex}
    \caption {An overview of the service worker being able to provide the application experience of both online and offline modes.}
    \label{fig:sw_interact}
\end{center}
\end{figure}

We are going to use a service worker in order to intercept the network traffic. For example, as can be seen at \figref*{fig:sw_interact}, in circumstances when problems are happening with a network when a request is in processing, we will get the data from the cache. At times, when the internet connection could not be established, the cache content can be easily displayed without forcing a user to wait, which dramatically improves performance and user experience. Also, when the internet connection is present, the data can be received from the actual network. 

\subsection{Service worker lifecycle}

In this subsection, we will introduce the steps through which a service worker goes. These steps are registration, installation, and activation. All of them we are going to describe below in more details.

\subsection*{Registration}

Before one can use service worker features, a developer has to register the corresponding script in the JavaScript code. The registration helps the browser to find a service worker and, afterwards, start its installation in the environment.

\begin{lstlisting}[caption={[Code for a Service Worker registration]Code, which demonstrates how to register a service worker\cite{32}.}, label={lst:tech1}]
if ('serviceWorker' in navigator) {
  navigator.serviceWorker.register('/service-worker.js')
  .then(function(registration) {
    console.log('Registration successful, scope is:', registration.scope);
  })
  .catch(function(error) {
    console.log('Service worker registration failed, error:', error);
  });
}
\end{lstlisting}

Looking at the \textit{line 1} of code in \lstref*{lst:tech1}, we see that, firstly, it is necessary to check, whether the browser supports service workers by observing the property \textit{serviceWorker} of \textit{navigator}\footnote{The navigator object contains information about the browser\cite{43}.}. If so, the service worker is going to be registered then, as can be seen at the \textit{line 2}, where the location of the service worker is stated as well. The \textit{navigator.serviceWorker.register} returns a promise\footnote{JavaScript Promise is an object, which represents the eventual completion or failure of an asynchronous operation\cite{37}.}, which resolves when the registration was successful. Afterwards, the scope of the service worker is logged with \textit{registration.scope}. 

Not every browser supports the functionality of service workers. Nowadays, the latest versions of such browsers as Chrome, Edge (partially) and Firefox support it. However, it is still only partially supported in Opera and Safari, and not at all in the Internet Explorer\cite{42} -- that is something developers should keep in mind.

The scope of the service workers is quite significant: it defines the paths, from which it can intercept network requests. By default, the scope of the service worker is the location, where the service worker file is stored, including all the sub-directories. For example, if the scope is the root directory, then the service worker is going to regulate the requests for all files at the domain.

\begin{lstlisting}[caption={[Code for setting up a custom scope of a Service Worker]Code, which demonstrates how to set a custom scope when registering a service worker\cite{32}.}, label={lst:tech2}]
navigator.serviceWorker.register('/service-worker.js', {
  scope: '/app/'
});
\end{lstlisting}

When registering, it is also possible to use a custom scope. \lstref*{lst:tech2} demonstrates that the service worker is going to have a scope of \textit{/app/}. It indicates that it will control requests from all the pages like \textit{/app/} and deeper. When the service worker is already installed, \textit{navigator.serviceWorker.register} returns the object of the currently active service worker.

\subsection*{Installation}

After the registration, the browser might attempt installation of a service worker, if it is considered as a new one, which happens in the following situations: 

\begin{itemize}
\item{when the site does not have a registered service worker yet;}
\item{when there is a difference between the previously installed service worker and the new one;}
\end{itemize}

\begin{lstlisting}[caption={[Code for listening to the installation of a Service Worker]Code, which demonstrates a listener for the \textit{install} service worker's event\cite{32}.}, label={lst:tech3}]
// Listen for install event, set callback
self.addEventListener('install', function(event) {
    // Perform some task
});
\end{lstlisting}

Installation triggers service worker's \textit{install} event. There is a probability of having a listener for it in order to assign a specific task (depends on the use case), which follows the installation on success. The way to set up this listener is shown in \lstref*{lst:tech3}.

\subsection*{Activation}

After the installation, a service worker has to be activated. In case there are open pages, which are controlled by the previous version of a service worker, then the recently installed service worker will be waiting. The activation of a new service worker takes place only when there are no other pages, controlled by the old version of a service worker. That provides a guarantee that only one version of service worker manages the pages of its scope at any time.

\begin{lstlisting}[caption={[Code for listening to activation of a Service Worker]Code, which demonstrates a listener for the \textit{activation} service worker's event\cite{32}.}, label={lst:tech4}]
self.addEventListener('activate', function(event) {
  // Perform some task
});
\end{lstlisting}

Similar to the installation, the activation phase also has its event that gets triggered once the service worker is activating, as we can see in \lstref*{lst:tech4}. For example, at this point, it might be a good idea to clean the previously stored old cached data.

\section{Cache API}
\label{CacheAPI}

For the best offline experience, the web application should store somewhere HTML, CSS, JavaScript code, as well as images, fonts. There is a place for all of it called Cache API, which we are going to use for WebCure. With Cache API it is possible to store network requests associated with corresponding requests.

\begin{lstlisting}[caption={[Code for creating cache storage using Cache API]Code, which demonstrates how one can create cache storage called \textit{my-cache}\cite{34}.}, label={lst:tech6}]
caches.open('my-cache').then((cache) => {
  // do something with cache...
});
\end{lstlisting}

In \lstref*{lst:tech6} we see an example code of how a cache with a name \textit{my-cache} is created. If the operation is performed successfully, then a promise resolves, and one is going to get access either to a newly created cache or to the one, which existed before the call of \textit{open} method.

That covers the main functionality of the API. After creating a cache, it is possible to manipulate it in many ways. 

Let us now introduce one of the operations -- adding elements to the cache. A developer can provide a string for the URL that will be fetched and stored in a cache object. Whenever the data is received from the cache, the browser will return a particular object according to the URL that was stored in the cache. More details on how long the data could be stored in the cache are given in \secref*{persistentstorage}.

Apart from adding to the cache object, it is also possible to remove stored URLs, check the current list of cached requests, delete a cache object, and other operations. 

\section{IndexedDB}
\label{IndexedDB}

As the protocol we introduced in \chapref*{Design} clearly describes that there is a necessity to have a client-side storage system, we are going to use for that purpose IndexedDB\cite{35}. It is a large-scale, NoSQL database, which lets us store any data in the user's browser. Apart from that, it supports transactions and achieves a high-performance search due to the usage of indexes. 

\subsection{IndexedDB terms}

In order to properly understand how IndexedDB works, it is quite useful to understand the concepts that are used in the database. First of all, each \textit{IndexedDB database} contains \textit{Object stores}. Those object stores, in its turn, are similar to tables in traditional databases. Usually, the practice is to have one object store for each type of data. This data could be anything: custom objects, strings, numbers, arrays, and other. It is possible to create more than one database, which could contain various object stores, but normally it is one database per application, which should have one object store for each type of data stored. To expedite the way of identifying objects in object stores, the latter have \textit{primary keys}, which must be unique in the particular object stores. Primary keys are defined by the developer and are very useful regarding searching the data.

All read and write operations in IndexedDB should be wrapped into a \textit{transaction}, which guarantees the database integrity. The critical point is that if one operation within a transaction fails, then none of the other operations are going to be applied. 

\subsection{IndexedDB Promised}

\begin{lstlisting}[caption={[Code for checking the support of IndexedDB API]Code, which demonstrates how one can check the support for IndexedDB API\cite{35}.}, label={lst:tech7}]
if (!('indexedDB' in window)) {
  console.log('This browser doesn\'t support IndexedDB');
  return;
}
\end{lstlisting}

As IndexedDB is relatively a new API, it is not supported by all the web browsers yet and, therefore, the support for it should be checked before any further development, as it is shown in \lstref*{lst:tech7}. However, all the recent versions of the major web browsers are compatible with it. 

Nevertheless, one of the most significant problems with IndexedDB is using it in the development. It has an asynchronous API, which is using events. That makes developers produce a complex code, which is hard to maintain. Therefore, in the design of WebCure, we are going to use a wrapper over the IndexedDB API -- IndexedDB Promised\cite{36, 48}. It is a tiny library, written by Jake Archibald from Google, which makes the use of JavaScript promises and simplifies the development process with IndexedDB while keeping its functionality.

\section{Persistent Storage}
\label{persistentstorage}

Both Cache API and IndexedDB database mentioned in \secref*{CacheAPI} and \secref*{IndexedDB} are taking place at the local machine. However, when the local machine is running out of storage space, user agents will clear it automatically on an LRU policy\footnote{In the systems with LRU or least recently used caching policy, the least frequently used data will be cleared first.}\cite{47}. This is not something that suits an application, which promises to provide offline experience. Moreover, in the worst case scenario, if the data was not synced with the server, it is going to be lost. The solution to this problem is to use a Persistent Storage API\cite{45, 46}, which guarantees that cached data is not going to be cleared if the browser comes under pressure.

\section{Background Sync}
\label{BackgroundSync}

The reality is that it is not always possible to be online all the time, even if someone wanted to. Sometimes, there is no network connection at all, or it could be that abysmal that it could be hard to do anything under such conditions. Therefore, the scenario when someone could do his or her work offline and then, when the connection is re-established again, this work will go online, is useful. Nowadays, thanks to \textit{Background Sync}\cite{38} from Google it is possible to do in web applications. 

\begin{lstlisting}[caption={[Code for registration of a sync event for a Service Worker]Code, which demonstrates how to register sync (\textit{myFirstSync} here) event for the service worker\cite{38}.}, label={lst:tech8}]
// Register the service worker:
navigator.serviceWorker.register('/sw.js');

// Then later, request a one-off sync:
navigator.serviceWorker.ready.then(function(swRegistration) {
  return swRegistration.sync.register('myFirstSync');
});
\end{lstlisting} 

For this feature to work, the web application has to use a service worker. First of all, it should be registered and, afterwards, a unique tag should be registered as well, which is going to be responsible for the background call of the method. Let us show an example: in \lstref*{lst:tech8}, we can see how a synchronisation tag \textit{myFirstSync} is registered.

\begin{lstlisting}[caption={[Code for calling a specific function when sync event triggered]Code, which demonstates that a function \textit{doSomeStuff} called, when the \textit{sync} event happened\cite{38}.}, label={lst:tech9}]
self.addEventListener('sync', function(event) {
  if (event.tag == 'myFirstSync') {
    event.waitUntil(doSomeStuff());
  }
});
\end{lstlisting} 

Afterwards, when a page controlled by the service worker is going back online (which is when the user agent has established a network connection\cite{44}), a \textit{sync} event is triggered. There, it is possible to perform a distinct action, depending on the registered earlier tag. For instance, in \lstref*{lst:tech9} we can see that a function \textit{doSomeStuff} is called once the connection is back, after a \textit{sync} event occurred. This function has to return a promise, which could be successful or not. In the latter case, another sync will be scheduled to try when there is a connectivity\cite{38}. 

Background synchronisation is an advantageous feature, which can be used in different scenarios. It could be, for example, sending the e-mails or any other type of messages, after having failed to send it when the connection was poor. However, in our case, in WebCure we are going to use it for sending the operations conducted offline back to the AntidoteDB server for further synchronisation. Moreover, if there was a use case with a requirement to get the most recent data from the server after the re-connection to the network, it is possible to implement it using the background synchronisation feature.