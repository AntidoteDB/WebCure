\chapter{Conclusion}
\label{Conclusion}

Conclusion

This is always the first chapter of the thesis. The chapter should be short (up to 5 pages). The chapter should feature sections as follows (where applicable): o Summary (Summarize this work in an insightful manner, assuming that the reader has seen the rest.) o Limitations or threats to validity (Point out the limitations of this work. In the case of empirical research, discuss threats to validity in a systematic manner.) o Future work (Provide insightful advice on where this research should be taken next.)

\section{Summary}
\section{Future Work}

- Automatic synchonization of updates;

\subsection*{Idempotence of updates}

- wrote about this part in the notes to the architecture pdf

In case a client is sending an update to the server and does not receive an acknowledgement, what could have happened?

There are possibilities:
\begin{itemize}
    \item {The update was applied on the server, but the connection failed when the acknowledgement was about to be sent;}
    \item {The update was not applied on the server and the client did not receive an acknowledgement;}
\end{itemize}
However, the client does not know which of these situations happened. Therefore, we have to find a general solution for such kind of behaviour. 

One of the solutions could be the following: it does not matter, whether the update was applied or not. A client will just send the update again, till it does not recieve an acknowledgement, regardless of what happens on the server's side.

The other one is: the client sends an update to the server, which should send a message back that it received an update. Afterwards, a client removes this update from the temporary database and sends back a message to the server that it is possible to apply the update.

However, the implementation of the solution for this problem is beyond the scope of this Master thesis. Therefore, these thoughts are going to be included in a section, where future impovements will be discussed.