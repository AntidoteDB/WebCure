% Included by MAIN.TEX
% Defines the settings for the CAMP report document

% use german characters as well
\usepackage[utf8]{inputenc}       % allow UTF-8 characters
%\usepackage[ansinew]{inputenc}       % allow ANSII-new characters

\renewcommand{\sectfont}{\normalfont \bfseries}        % Schriftart der Kopfzeile

% manipulate footer
\usepackage{scrpage2}
\pagestyle{scrheadings}
\ifoot[\footertext]{\footertext} % \footertext set in INFO.TEX
%\setkomafont{pagehead}{\normalfont\rmfamily}
\setkomafont{pagenumber}{\normalfont\rmfamily}


%\usepackage{algorithm}
%\usepackage{algpseudocode}
%\floatname{algorithm}{Algorithmus}
%\usepackage{framed}
%\usepackage{multicol}

%% allow sophisticated control structures
\usepackage{ifthen}

% use Palatino as default font
\usepackage{palatino}

% enable special PostScript fonts
\usepackage{pifont}

%to use the subfigures
\usepackage{subfig}

%\usepackage{wrapfig}
\usepackage{listings}
\usepackage{color}
\usepackage[svgnames]{xcolor}
\usepackage{colortbl}

% for js listings
\usepackage{caption}
\usepackage{jslistings}

\lstset{aboveskip=20pt,belowskip=20pt}

\usepackage{url}
\usepackage{stackengine}

\usepackage{xparse,nameref}
\NewDocumentCommand{\chapref}{s m}{\textbf{Chapter\textcolor{blue}{~\ref{#2}}}\IfBooleanF{#1}{ \nameref{#2}}}
\NewDocumentCommand{\figref}{s m}{\textbf{Figure\textcolor{blue}{~\ref{#2}}}\IfBooleanF{#1}{ \nameref{#2}}}
\NewDocumentCommand{\lstref}{s m}{\textbf{Listing\textcolor{blue}{~\ref{#2}}}\IfBooleanF{#1}{ \nameref{#2}}}


%\makeatletter 
%\g@addto@macro\UrlBreaks{ 
%  \do\a\do\b\do\c\do\d\do\e\do\f\do\g\do\h\do\i\do\j 
%  \do\k\do\l\do\m\do\n\do\o\do\p\do\q\do\r\do\s\do\t 
%  \do\u\do\v\do\w\do\x\do\y\do\z\do\&\do\1\do\2\do\3 
%  \do\4\do\5\do\6\do\7\do\8\do\9\do\0} 
%% \def\do@url@hyp{\do\-} 
%\makeatother 

\usepackage{url}
\urlstyle{same}

%% show program code\ldots
%\usepackage{verbatim}
%\usepackage{program}

%% enable TUM symbols on title page
\usepackage{styles/tumlogo}


\usepackage{multirow}

%% use colors
\usepackage{color}

%% make fancy math
\usepackage{amsmath}
\usepackage{amsfonts}
\usepackage{amssymb}
\usepackage{textcomp}
\usepackage{yhmath} % f�r die adots 
%% mark text as preliminary
%\usepackage[draft,german,scrtime]{prelim2e}

%% create an index
\usepackage{makeidx}

% for the program environment
%\usepackage{float}
\usepackage{tocbasic}

%% load german babel package for german abstract
%\usepackage[german,american]{babel}
\usepackage[ngerman]{babel}
%\selectlanguage{german}

% use initals dropped caps - doesn't work with PDF
%\usepackage{dropping}


\usepackage{styles/shortoverview}
%----------------------------------------------------
%      Graphics and Hyperlinks
%----------------------------------------------------


%% check for pdfTeX
\ifx\pdftexversion\undefined
 %% use PostScript graphics
 \usepackage[dvips]{graphicx}
 \DeclareGraphicsExtensions{.eps,.epsi}
 \graphicspath{{figures/}{figures/review}} 
 %% allow rotations
 \usepackage{rotating}
 %% mark pages as draft copies
 %\usepackage[english,all,light]{draftcopy}
 %% use hypertex version of hyperref
 \usepackage[hypertex,hyperindex=false,colorlinks=false,breaklinks]{hyperref} 
 %% breaks urls if they are too long
 \usepackage{breakurl}
\else %% reduce output size \pdfcompresslevel=9
 %% declare pdfinfo
 %\pdfinfo { 
 %  /Title (my title) 
 %  /Creator (pdfLaTeX) 
 %  /\dfrac{Author}{den} (my name) 
 %  /Subject (my subject	) 
 %  /Keywords (my keywords)
 %}
 %% use pdf or jpg graphics
 \usepackage[pdftex]{graphicx}
 \DeclareGraphicsExtensions{.jpg,.JPG,.png,.pdf,.eps}
 \graphicspath{{figures/}} 
 
 %% allow rotations
 \usepackage{rotating}
 %% use pdftex version of hyperref
 \usepackage[pdftex,colorlinks=true,linkcolor=blue,citecolor=black,%
 anchorcolor=black,urlcolor=black,bookmarks=true,%
 bookmarksopen=true,bookmarksopenlevel=0,plainpages=false,%
 bookmarksnumbered=true,hyperindex=false,pdfstartview=%
 ]{hyperref}
%
%\usepackage[pdftex,colorlinks=false,linkcolor=red,citecolor=red,%
% anchorcolor=red,urlcolor=red,bookmarks=true,%
% bookmarksopen=true,bookmarksopenlevel=0,plainpages=false%
% bookmarksnumbered=true,hyperindex=false,pdfstartview=%
% ]{hyperref}




%% Fancy chapters
%\usepackage[Lenny]{fncychap}
%\usepackage[Glenn]{fncychap}
%\usepackage[Bjarne]{fncychap}

%\usepackage[avantgarde]{quotchap}

% set the bibliography style
\bibliographystyle{unsrtnat}

\usepackage{textcomp}
%\definecolor{listinggray}{gray}{0.9}
%\definecolor{lbcolor}{rgb}{0.9,0.9,0.9}
%\lstset{language=XML}

\usepackage[all]{hypcap}
\pdfminorversion=6

\usepackage[T1]{fontenc}
\usepackage{epsfig}
\usepackage[numbers,sort&compress]{natbib}
\usepackage{bm}
\usepackage[normalem]{ulem}

%% Do tabs
\usepackage{tabto}

\addtokomafont{captionlabel}{\bfseries}

%% Use enumeration of subsubsections
\setcounter{secnumdepth}{3}

%% Use footnote numbering for whole document
\usepackage{chngcntr} 
\counterwithout{footnote}{chapter}

\DeclareNewTOC[
    counterwithin=chapter,
    float, type=codelisting, types=codelistings,
    name={Listing}, listname={List of Code}
]{loc}
\setuptoc{loc}{chapteratlist}

%% \DeclareNewTOC[%
%	counterwithin=chapter,%
%	hang=2em,%
%	type=example,%
%	types=examples,%
%	nonfloat,%
%	name=Beispiel,%
%	listname={Beispielverzeichnis}%
%]{loex}
%\setuptoc{loex}{chapteratlist}

%\DeclareNewTOC[%
%	counterwithin=chapter,%
%	hang=2em,%
%	type=definition,%
%	types=definitions,%
%	nonfloat,%
%	name=Begriff,%
%	listname={Begriffsverzeichnis}%
%]{lode}
%\setuptoc{lode}{chapteratlist}

%\setlength{\textfloatsep}{-20pt}
%\setlength{\belowcaptionskip}{10pt plus 0pt minus 0pt}